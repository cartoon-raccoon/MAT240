\documentclass[12pt, a4paper]{article}
\setlength{\parindent}{0pt}
\usepackage[a4paper, portrait, margin=1in]{geometry}
\usepackage{preamble}

\newtheorem*{theorem}{Theorem}

\newcommand{\F}{\mathbb{F}}
\newcommand{\im}{\operatorname{im}}

\begin{document}
\textbf{(Q9)}

\textit{(a)}
The rotation on $P_{xy}$ is given as
\[
    T(x, y, z) = (x\cos\theta - y\sin\theta,
    x\sin\theta + y\cos\theta, z)
\]
Since for any $v = (x, y, 0) \in P_{xy}$, $T(v)$ is still of the form
$(x, y, 0)$, $T(P_{xy}) = P_{xy}$.

Similarly, for any $(0, 0, z)$ on the $z$-axis, rotation gives the
same form $(0, 0, z)$, this the line $(0, 0, z)$ is also T-invariant.

\textit{(b)}
Taking $T$ and splitting by parameter, we have
\[
    x(\cos\theta + \sin\theta + 0)+
    y(-\sin\theta + \cos\theta + 0)+
    z(0, 0, 1)
\]
which yields the matrix
\[
    \begin{bmatrix}
        \cos\theta & -\sin\theta & 0 \\
        \sin\theta &  \cos\theta & 0 \\
             0     &       0     & 1
    \end{bmatrix}
\]
\end{document}