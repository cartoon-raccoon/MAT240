\documentclass[12pt, a4paper]{article}
\setlength{\parindent}{0pt}
\usepackage[a4paper, portrait, margin=1in]{geometry}
\usepackage{preamble}

\newtheorem*{theorem}{Theorem}

\newcommand{\F}{\mathbb{F}}
\newcommand{\im}{\operatorname{im}}

\begin{document}
\textbf{(Q5)}

\textit{(a)}
Each vector $\mathbf{a}_i$ in $\gamma$ is given by $A\mathbf{e}_i$, so
\[
    A = \left[A\mathbf{e}_1, \ldots A\mathbf{e}_n\right] =
    \left[T_A(\mathbf{e}_1), \ldots T_A(\mathbf{e}_n)\right] =
    \left[I_{\F^n}\right]_\gamma^\beta
\]
Since $A$ is a change of basis matrix, it must be invertible.

\textit{(b)}
\textit{(i)}
Since the columns of $A$ are given by the elements of $\gamma$:
\[
    A_i = \mathbf{a}_i = T_A(\beta_i)
\]
any vector in $\im T_A$ can be expressed as a linear combination
of the columns of $A$, which is $\gamma$.

\end{document}