\documentclass[12pt, a4paper]{article}
\setlength{\parindent}{0pt}
\usepackage[a4paper, portrait, margin=1in]{geometry}
\usepackage{preamble}

\newtheorem*{theorem}{Theorem}

\newcommand{\F}{\mathbb{F}}
\newcommand{\im}{\operatorname{im}}

\begin{document}
\textbf{(Q1)}
We first row-reduce normally:
\begin{gather*}
    \begin{pmatrix}
        1 & 2 & -1 & 2 \\
        1 & 1 &  0 & 2 \\
        1 & 3 & -2 & 2
    \end{pmatrix}\\
    \downarrow R2 - R1\\
    \begin{pmatrix}
        1 & 2 & -1 & 2 \\
        0 & -1 & 1 & 0 \\
        1 & 3 & -2 & 2
    \end{pmatrix}\\
    \downarrow R3 - R1\\
    \begin{pmatrix}
        1 & 2 & -1 & 2 \\
        0 & -1 & 1 & 0 \\
        0 & 1 & -1 & 0
    \end{pmatrix}\\
    \downarrow R2 \cdot -1\\
    \begin{pmatrix}
        1 & 2 & -1 & 2 \\
        0 & 1 & -1 & 0 \\
        0 & 1 & -1 & 0
    \end{pmatrix}\\
    \downarrow R3 - R2\\
    \begin{pmatrix}
        1 & 2 & -1 & 2 \\
        0 & 1 & -1 & 0 \\
        0 & 0 &  0 & 0
    \end{pmatrix}\\
    \downarrow R1 - 2R2\\
    \begin{pmatrix}
        1 & 0 &  2 & 2 \\
        0 & 1 & -1 & 0 \\
        0 & 0 &  0 & 0
    \end{pmatrix}\\
\end{gather*}

We can then take each elementary matrix in reverse order of RREF operations
and multiply them together to get the matrix
\[
    \begin{pmatrix}
        -1 &  2 & 0 \\
        1  & -1 & 0 \\
        -2 &  1 & 0
    \end{pmatrix}
\]
\end{document}