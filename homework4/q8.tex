\documentclass[12pt, a4paper]{article}
\setlength{\parindent}{0pt}
\usepackage[a4paper, portrait, margin=1in]{geometry}
\usepackage{preamble}

\newtheorem*{theorem}{Theorem}

\newcommand{\F}{\mathbb{F}}
\newcommand{\im}{\operatorname{im}}
\newcommand{\rank}{\operatorname{rank}}

\begin{document}
\textbf{(Q8)}

\textit{(a)}
$\{(x, mx) | x \in \R\}$ and $\{(x, -\tfrac{1}{m}x) | x \in \R\}$.

$y = -\frac{1}{m}x$ is orthogonal to $y = mx$, so any reflection across
the latter of any element in $y = -\frac{1}{m}$ would still lie on the
same line.

\textit{(b)}
\begin{proof}
    By earlier proof, $T$ is an isomorphism, so $N(T) = \{0_{\R^3}\}$.
    Then $T(N(T))$ = $N(T)$ by linearity, so $N(T)$ is T-invariant.

    Since $T$ is isomorphic, 
    \[
        \rank T = \dim T = \dim (\im T) = n \implies
        \im T = \R^3 \sse \R^3
    \]
\end{proof}
\end{document}