\documentclass[12pt, a4paper]{article}
\setlength{\parindent}{0pt}
\usepackage[a4paper, portrait, margin=1in]{geometry}
\usepackage{preamble}

\newtheorem*{theorem}{Theorem}

\newcommand{\F}{\mathbb{F}}
\newcommand{\im}{\operatorname{im}}

\begin{document}
\textbf{(Q6)}

\begin{proof}
    First, suppose $T(\beta)$ is a basis for $W$. Let
    $\beta = \{\mathbf{v}_1, \mathbf{v}_2, \ldots \mathbf{v}_n\}$. Then
    $T(\beta) = \{T(v_1), T(v_2), \ldots T(v_n)\}$.
    Since $T$ is linear and $|\beta| = |T(\beta)| \implies \dim V = \dim W$,
    $T$ is an isomorphism.
    
    Then suppose $T$ is an isomorphism. Thus $T$ is injective.

    Since $T$ is injective, every element in $T(\beta)$ is distinct, and
    $W$ spans $T(\beta)$
    Then to prove linear independence, we express the elements of
    $T(\beta)$ as a linear combination equating to 0:
    \[
        a_1T(v_1) + a_2T(v_2) + \ldots + a_nT(v_n) = 0
    \]
    Then by linearity and the properties of the null space:
    \begin{align*}
        a_1T(v_1) + \ldots + a_nT(v_n) & = T(a_1v_1) + \ldots + T(a_nv_n)\\
        & = T(a_1v_1 + \ldots + a_nv_n)\\
        & = 0_W
    \end{align*}
    Since $T$ is injective, $N(T) = \{0_V\}$ and thus
    \[
        a_1v_1 + a_2v_2 + \ldots + a_nv_n = 0_V
    \]
    Since $\beta$ is a basis, it is linearly independent and thus
    all the coefficients are also $0$, thus $T(\beta)$ is also
    linearly independent.
\end{proof}
\end{document}