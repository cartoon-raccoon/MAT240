\documentclass[12pt, a4paper]{article}
\setlength{\parindent}{0pt}
\usepackage[a4paper, portrait, margin=1in]{geometry}
\usepackage{preamble}

\newtheorem*{theorem}{Theorem}

\newcommand{\F}{\mathbb{F}}
\newcommand{\im}{\operatorname{im}}

\begin{document}
\textbf{(Q2)}

\textit{(a)}
\begin{proof}
    For $n = 1$, there is nothing to prove.

    For $n = 2$, we have the matrix
    \[
        A = \begin{pmatrix}
            a & b\\
            a & b
        \end{pmatrix}
    \]
    Then $\det A = ab - ab = 0$.
    
    For the induction step, we assume this holds for $n$, and prove
    for $n + 1$.

    Let $A \in \mathcal{M}_{n + 1 \times n + 1}$, and $A$ has two
    identical rows. Using Part 2 of Theorem 2, we define $\det A$ as
    \[
        \sum_{j = 1}^{n + 1}(-1)^{i + j}A_{ij}
        \det\left(\tilde{A}_{ij}\right)
    \]
    Where $i \in \{1, 2, \ldots, n\}$. We choose $i$ such that we are not
    choosing one of the two identical rows.
    Then $\tilde{A}_{ij} \in \mathcal{M}_{n \times n}$ and has two
    identical rows, so $\det (\tilde{A}_{ij}) = 0$.

    Expanding the summation for $\det A$, we then get:
    \[
        A_{i1}\det(\tilde{A}_{i1}) - A_{i2}\det(\tilde{A}_{i2}) + \ldots
        A_{i(n + 1)}\det(\tilde{A}_{i(n+1)})
    \]
    Since $\det (\tilde{A}_{ij}) = 0$, each term in this sum is 0,
    and thus the entire sum is 0.
\end{proof}

\newpage
\textit{(b)}
\begin{proof}
    Let
    \[
        A = \begin{pmatrix}
            a_1\\ \vdots\\
            a_i\\ \vdots\\
            a_j\\ \vdots\\
            a_n
        \end{pmatrix}, \quad
        B = \begin{pmatrix}
            a_1\\ \vdots\\
            a_j\\ \vdots\\
            a_i\\ \vdots\\
            a_n
        \end{pmatrix},
    \]
    By Theorem 57, we observe that
    \begin{gather*}
        \det C = \det \begin{pmatrix}
            a_1\\ \vdots\\
            a_i + a_j\\ \vdots\\
            a_j + a_i\\ \vdots\\
            a_n
        \end{pmatrix} =
        \det \begin{pmatrix}
            a_1\\ \vdots\\
            a_i\\ \vdots\\
            a_j + a_i\\ \vdots\\
            a_n
        \end{pmatrix} +
        \det \begin{pmatrix}
            a_1\\ \vdots\\
            a_j\\ \vdots\\
            a_j + a_i\\ \vdots\\
            a_n
        \end{pmatrix}\\
        = \det \begin{pmatrix}
            a_1\\ \vdots\\
            a_i\\ \vdots\\
            a_j\\ \vdots\\
            a_n
        \end{pmatrix} +
        \det \begin{pmatrix}
            a_1\\ \vdots\\
            a_i\\ \vdots\\
            a_i\\ \vdots\\
            a_n
        \end{pmatrix} +
        \det \begin{pmatrix}
            a_1\\ \vdots\\
            a_j\\ \vdots\\
            a_i\\ \vdots\\
            a_n
        \end{pmatrix} +
        \det \begin{pmatrix}
            a_1\\ \vdots\\
            a_j\\ \vdots\\
            a_j\\ \vdots\\
            a_n
        \end{pmatrix}\\
        = \det A + 0 + \det B + 0
    \end{gather*}
    Since $C$ has two identical rows, $a_i + a_j$ and $a_j + a_i$,
    By Theorem 58 part 1, $\det C = \det A + \det B = 0$.
    Thus, we can conclude
    \[
        \det A + \det B = 0 \implies \det A = - \det B
    \]
\end{proof}
\end{document}