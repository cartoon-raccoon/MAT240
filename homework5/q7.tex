\documentclass[12pt, a4paper]{article}
\setlength{\parindent}{0pt}
\usepackage[a4paper, portrait, margin=1in]{geometry}
\usepackage{preamble}

\newtheorem*{theorem}{Theorem}

\newcommand{\F}{\mathbb{F}}
\newcommand{\im}{\operatorname{im}}

\begin{document}
\textbf{(Q7)}

1. If $A$ has a column of zeroes, then $\det A = 0$.
\begin{proof}
    Let $A$ have a column $a_i$ where $a_{1i} = a_{2i} = \ldots a_{ni}= 0$.
    Then $A^t$ has $a_{i1} = a_{i2} = \ldots a_{in}= 0$, i.e a row
    of zeroes, so $\det A^t = 0$ by Theorem 58 Part 1.

    So $\det A^t = \det A = 0$.
\end{proof}

2. For any $i \in \{1,\ldots,n\}$, we have
\[
    \det A = \sum_{i = 1}^{n} (-1)^{i + j} A_{ij} \det (\tilde{A}_{ij})
\]
\begin{proof}
    The proof for this follows from Q3.
\end{proof}

3. If $A$ has two identical columns, then $\det A = 0$.
\begin{proof}
    Let the columns of $A$ be given by
    \[
        \begin{pmatrix}
            a_1 & \dots & a_i & \dots & a_i \dots & a_n
        \end{pmatrix}
    \]
    Then for $A^t$, Each row is given by
    \[
        [A^t]_{ij} = A_{ji}
    \]
    So $A^t$ has two identical rows. Thus
    \[
        \det (A^t) = \det A = 0
    \]
\end{proof}

4. If $B$ is obtained from $A$ by swapping two columns, then
$\det B = -\det A$.
\begin{proof}
    Let $A = \left(a_{n1}, \dots, a_{ni}, \dots a_{nj}, \dots, a_{nn}\right)$,
    and $B = \left(a_{n1}, \dots, a_{nj}, \dots a_{ni}, \dots, a_{nn}\right)$

    Then for each element in $A^t$ and $B^t$:
    \[
        [A^t]_{ij} = A_{ji}, [A^t]_{ij} = B_{ji}
    \]
    So now $B^t$ is obtained from $A^t$ by swapping two rows.
    $\det A^t = - \det B^t$ by Q2(b).

    Then 
    \[
        \det A^t = -\det B^t = -\det B = \det A
    \]
\end{proof}

5. If $B$ is obtained from $A$ by scaling column $i$ by $c \in \F$, then
$\det B = a \det A$.
\begin{proof}
    Let $B = \left(a_{n1}, \dots ca_{ni}, \dots, a_{nn}\right)$.
    Then $[B^t]_{ij} = B_{ji}$, and $[A^t]_{ij} = A_{ji}$,
    So $B^t$ is obtained from $A^t$ by scaling row $i$ of $A^t$
    by $c$.

    Then
    \[
        \det B^t = c \det A^t = c \det A = \det B
    \]
\end{proof}

\newpage
6. If $B$ is obtained from $A$ by adding a multiple of one column to another,
then $\det B = \det A$.
\begin{proof}
    Let 
    \[
        A = \left(a_{n1}, \dots, a_{ni}, \dots a_{nj}, \dots, a_{nn}\right)
    \]
    and
    \[
        B = \left(a_{n1}, \dots, a_{ni} + a_{nj}, \dots, a_{nj}, \dots, a_{nn}\right)
    \]

    Then
    \[
        A^t = \begin{pmatrix}
            a_{1n}\\ \vdots\\
            a_{in}\\ \vdots\\
            a_{jn}\\ \vdots\\
            a_{nn}
        \end{pmatrix}
        \quad
        B^t = \begin{pmatrix}
            a_{1n}\\ \vdots\\
            a_{in} + a_{jn}\\ \vdots\\
            a_{jn}\\ \vdots\\
            a_{nn}
        \end{pmatrix}
    \]
    Then
    \[
        \det A^t = \det B^t = \det A = \det B
    \]
\end{proof}
\end{document}