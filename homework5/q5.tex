\documentclass[12pt, a4paper]{article}
\setlength{\parindent}{0pt}
\usepackage[a4paper, portrait, margin=1in]{geometry}
\usepackage{preamble}

\newtheorem*{theorem}{Theorem}

\newcommand{\F}{\mathbb{F}}
\newcommand{\im}{\operatorname{im}}

\begin{document}
\textbf{(Q5)}
\begin{proof}
    We prove this with induction on $n$.

    For $n = 1$, there is nothing to prove.

    For $n = 2$, let $A = \begin{pmatrix}a & b \\ 0 & c\end{pmatrix}$.
    $\det A$ is then $ac - 0b = ac$.

    For the inductive step, we assume this holds for an upper triangular
    matrix $A \in \mathcal{M}_{n \times n}$ and prove for $n + 1$.

    Let $A \in \mathcal{M}_{(n + 1) \times (n + 1)}$.
    We expand $\det A$ with "row-$i$" expansion on row $n + 1$, the last row.

    This gives us
    \[
        \det A = \sum_{j = 1}^{n + 1} (-1)^{(n + 1) + j}\det (\tilde{A}_{nj})
    \]
    Since every element in the last row is 0 except for element $A_{(n+1)(n+1)}$,
    we can simplify this to:
    \[
        (-1)^{2(n + 1)} A_{(n + 1)(n + 1)} \det (\tilde{A}_{nn})
        \quad\quad (1)
    \]
    Since $(-1)^{2n}$ is always positive, it disappears. We also observe
    that $\tilde{A}_{nn}$ is an upper triangular matrix of $n \times n$,
    so we can assume $\det (\tilde{A}_{nn})$ is a product of its diagonal
    entries.
    Since $A_{(n + 1)(n + 1)}$ is the last diagonal entry of $A$, by (1),
    $\det A$ is also the product of its diagonal entries.
\end{proof}
\end{document}