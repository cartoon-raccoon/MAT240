\documentclass[12pt, a4paper]{article}
\setlength{\parindent}{0pt}
\usepackage[a4paper, portrait, margin=1in]{geometry}
\usepackage{preamble}

\newtheorem*{theorem}{Theorem}

\newcommand{\F}{\mathbb{F}}
\newcommand{\im}{\operatorname{im}}

\begin{document}
\textbf{(Q11)}

\textit{(a)} True.

\begin{proof}
    Let $A \sim B$. Then $A = PBP^{-1}$, where $P \in \mathcal{M}_{n \times n}$.
    Then
    \[
        A^k = (PBP^{-1})^k = \underbrace{
            (PBP^{-1})(PBP^{-1}) \ldots (PBP^{-1})
        }_{k \text{ times}}
    \]

    Since between each occurrence of $B$, we have $P^{-1}P$, they cancel into $I_n$
    which then disappears, leaving us with
    \[
        P\underbrace{BB \ldots B}_{k \text{ times}}P^{-1} = PB^kP^{-1}
    \]
    So $A^k = PB^kP^{-1} \iff A^k \sim B^k$.
\end{proof}

\textit{(b)} False.

As a counterexample, let
\[
    A = \begin{pmatrix}
        0 & 1\\
        0 & 0
    \end{pmatrix}\quad
    B = \begin{pmatrix}
        0 & 0\\
        0 & 0
    \end{pmatrix}
\]
$A^2 = B^2$, but $A$ and $B$ are clearly not similar.

\textit{(c)} False.

As a counterexample, let
\[
    A = \begin{pmatrix}
        5 & 7\\
        2 & 3
    \end{pmatrix}\quad
    B = \begin{pmatrix}
        1 & 0\\
        0 & 1
    \end{pmatrix}
\]
$\det (AB) = 1$, but A and B are clearly not inverses of each other.
\end{document}