\documentclass[12pt, a4paper]{article}
\setlength{\parindent}{0pt}
\usepackage[a4paper, portrait, margin=1in]{geometry}
\usepackage{preamble}

\newtheorem*{theorem}{Theorem}

\newcommand{\F}{\mathbb{F}}
\newcommand{\im}{\operatorname{im}}

\begin{document}
\textbf{(Q8)}

\textit{(a)} We find $C_A(x)$ by computing $\det(xI_n - A)$:
\[
    B = \begin{pmatrix}
        x & 0 & 0 & 0\\
        0 & x & 0 & 0\\
        0 & 0 & x & 0\\
        0 & 0 & 0 & x
    \end{pmatrix} -
    \begin{pmatrix}
        1 & 0 & 1 & 1\\
        0 & 1 & 0 & 0\\
        0 & 0 & 3 & 2\\
        0 & 0 &-1 & 0
    \end{pmatrix} =
    \begin{pmatrix}
        x-1 & 0 &-1 &-1\\
        0 & x-1 & 0 & 0\\
        0 & 0 & x-3 &-2\\
        0 & 0 & -1 & x
    \end{pmatrix}
\]
\[
    \det B = (x - 1)^2\left(x(x-3) + 2\right)
    =(x - 1)^3(x - 2)
\]
Since $C_A(x)$ splits over $\R$, $A$ has eigenvalues 1 and 2.

\textit{(b)} For each eigenvalue $\lambda$, we solve for is eigenspace
$E_\lambda = N(\lambda I_n - A)$ and find a basis for $E_\lambda$.

For $E_1$:
\[
    I_4 - A = \begin{pmatrix}
        0 & 0 & -1 & -1\\
        0 & 0 & 0 & 0\\
        0 & 0 & 2 & 2\\
        0 & 0 & 1 & 1
    \end{pmatrix} \xrightarrow{RREF}
    \begin{pmatrix}
        0 & 0 & 1 & 1\\
        0 & 0 & 0 & 0\\
        0 & 0 & 0 & 0\\
        0 & 0 & 0 & 0
    \end{pmatrix}
\]
Solving the general solution, we get the basis
\[
    \beta_{E_1} = \{
        (1, 0, 0, 0),
        (0, 1, 0, 0),
        (0, 0, -1, 1)
    \}
\]

We then do the same for $E_2$:
\[
    2I_4 - A = \begin{pmatrix}
        1 & 0 & -1 & -1\\
        0 & 1 & 0 & 0\\
        0 & 0 & -1 & -2\\
        0 & 0 & 1 & 2
    \end{pmatrix} \xrightarrow{RREF}
    \begin{pmatrix}
        1 & 0 & 0 & 1\\
        0 & 1 & 0 & 0\\
        0 & 0 & 1 & 2\\
        0 & 0 & 0 & 0
    \end{pmatrix}
\]
Solving the general solution, we get the basis
\[
    \beta_{E_2} = \{
        (-1, 0, -2, 1)
    \}
\]

\textit{(c)} The union $\beta_1 \cup \beta_2$ is given by:
\[
    \beta = \{
        (1, 0, 0, 0),
        (0, 1, 0, 0),
        (0, 0, -1, 1)
        (-1, 0, -2, 1)
    \}
\]
$\beta$ spans $\R^4$, and is linearly independent. Thus, $\beta$
is a basis for $\R^4$.

\textit{(d)} Since $\R^4$ is a direct sum of the eigenspaces of $A$,
$A$ is diagonalizable. To do so, we arrange the vectors of $\beta$
to the columns of a matrix:
\[
    P = \begin{pmatrix}
        1 & 0 & 0 & 1 \\
        0 & 1 & 0 & 0 \\
        0 & 0 &-1 &-2 \\
        0 & 0 & 1 & 1
    \end{pmatrix}
\]

\newpage
We then solve for $P^{-1}$:
\begin{gather*}
    \left(\begin{array}{cccc|cccc}
        1 & 0 & 0 & 1 & 1 & 0 & 0 & 0 \\
        0 & 1 & 0 & 0 & 0 & 1 & 0 & 0 \\
        0 & 0 &-1 &-2 & 0 & 0 & 1 & 0 \\
        0 & 0 & 1 & 1 & 0 & 0 & 0 & 1
    \end{array}\right) \xrightarrow{R3 \cdot -1}
    \left(\begin{array}{cccc|cccc}
        1 & 0 & 0 & 1 & 1 & 0 & 0 & 0 \\
        0 & 1 & 0 & 0 & 0 & 1 & 0 & 0 \\
        0 & 0 & 1 & 2 & 0 & 0 &-1 & 0 \\
        0 & 0 & 1 & 1 & 0 & 0 & 0 & 1
    \end{array}\right) \\ \xrightarrow{R4 - R3}
    \left(\begin{array}{cccc|cccc}
        1 & 0 & 0 & 1 & 1 & 0 & 0 & 0 \\
        0 & 1 & 0 & 0 & 0 & 1 & 0 & 0 \\
        0 & 0 & 1 & 2 & 0 & 0 &-1 & 0 \\
        0 & 0 & 0 &-1 & 0 & 0 & 1 & 1
    \end{array}\right) \xrightarrow{R4 \cdot -1}
    \left(\begin{array}{cccc|cccc}
        1 & 0 & 0 & 1 & 1 & 0 & 0 & 0 \\
        0 & 1 & 0 & 0 & 0 & 1 & 0 & 0 \\
        0 & 0 & 1 & 2 & 0 & 0 &-1 & 0 \\
        0 & 0 & 0 & 1 & 0 & 0 &-1 &-1
    \end{array}\right) \\ \xrightarrow{R1 + R4, R3 - 2R4}
    \left(\begin{array}{cccc|cccc}
        1 & 0 & 0 & 0 & 1 & 0 &-1 &-1 \\
        0 & 1 & 0 & 0 & 0 & 1 & 0 & 0 \\
        0 & 0 & 1 & 0 & 0 & 0 & 1 & 2 \\
        0 & 0 & 0 & 1 & 0 & 0 &-1 &-1
    \end{array}\right)
\end{gather*}
We then solve for $D = P^{-1}AP$:
\[
    \begin{pmatrix}
        1 & 0 &-1 &-1 \\
        0 & 1 & 0 & 0 \\
        0 & 0 & 1 & 0 \\
        0 & 0 & 0 & 1
    \end{pmatrix} \cdot
    \begin{pmatrix}
        1 & 0 & 1 & 1\\
        0 & 1 & 0 & 0\\
        0 & 0 & 3 & 2\\
        0 & 0 &-1 & 0
    \end{pmatrix} \cdot
    \begin{pmatrix}
        1 & 0 & 0 & 1 \\
        0 & 1 & 0 & 0 \\
        0 & 0 &-1 &-2 \\
        0 & 0 & 1 & 1
    \end{pmatrix}
\]
Which yields the diagonal matrix
\[
    \begin{pmatrix}
        1 & 0 & 0 & 0 \\
        0 & 1 & 0 & 0 \\
        0 & 0 & 1 & 0 \\
        0 & 0 & 0 & 2
    \end{pmatrix}
\]
\end{document}