\documentclass[12pt, a4paper]{article}
\setlength{\parindent}{0pt}
\usepackage[a4paper, portrait, margin=1in]{geometry}
\usepackage{preamble}

\newtheorem*{theorem}{Theorem}

\newcommand{\F}{\mathbb{F}}
\newcommand{\im}{\operatorname{im}}

\begin{document}
\textbf{(Q1)}
\begin{theorem}
    If a matrix $A \in \mathcal{M}_{n \times n}$ has a zero row, then
    $\det A = 0$.
\end{theorem}
\begin{proof}
    We prove this by induction on $n$.

    First we consider $n = 1$. We then have a matrix $A = (0)$, whose
    determinant is clearly 0.

    We then consider $n = 2$. We construct a matrix
    \[
        A = \begin{pmatrix}
            0 & 0 \\
            a & b
        \end{pmatrix}
    \]
    $\det A$ is then $0b - 0a = 0$.

    We then assume $\det A$ for $A \in \mathcal{M}_{n \times n}$ is 0, and
    prove this also holds for the $n + 1$ case.

    The determinant of $A \in \mathcal{M}_{n + 1 \times n + 1}$ is given by:
    \[
        \sum_{j = 1}^{n + 1} (-1)^{1 + j} A_{1j} \det (\tilde{A}_{1j})
    \]

    We observe that $\det (\tilde{A}_{1j})$ is the determinant of an $n \times n$
    matrix, so it will be 0 if it has a zero row.

    We now consider two possibilities:

    \textit{Case 1:} The zero row of the matrix is the top row.

    In this case, the term $A_{1j}$ will always be 0.
    The determinant will then evaluate to:

    \begin{align*}
        & \sum_{j = 1}^{n + 1} (-1)^{1 + j} A_{1j} \det (\tilde{A}_{1j})\\
        & = \sum_{j = 1}^{n + 1} (-1)^{1 + j} 0 \det (\tilde{A}_{1j})\\
        & = 0
    \end{align*}

    \textit{Case 2:} The zero row of the matrix is from row 2 onwards.

    In this case, the matrix $\tilde{A}_{1j} \in \mathcal{M}_{n \times n}$ 
    will have a zero row,
    so $\det \tilde{A}_{1j} = 0$.

    Then $\det A$ evaluates to:

    \begin{align*}
        & \sum_{j = 1}^{n + 1} (-1)^{1 + j} A_{1j} \det (\tilde{A}_{1j})\\
        & = \sum_{j = 1}^{n + 1} (-1)^{1 + j} A_{1j} 0\\
        & = 0
    \end{align*}
\end{proof}
\end{document}