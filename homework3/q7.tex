\documentclass[12pt, a4paper]{article}
\setlength{\parindent}{0pt}
\usepackage[a4paper, portrait, margin=1in]{geometry}
\usepackage{preamble}

\newtheorem*{theorem}{Theorem}

\newcommand{\F}{\mathbb{F}}
\newcommand{\im}{\operatorname{im}}

\begin{document}
\textbf{(Q7)}

\textit{(a)} False.

The contrapositive of the given statement is that if $\dim V \geq \dim W$,
then there exists at least one injective linear map. This means
that there exists one linear map $T$ such that $N(T) = \{\mathbf{0}\}$.

By the Rank-Nullity Theorem:
\begin{align*}
    \dim (\im T) + \dim N(T) = \dim V & \implies 
    \dim (\im T) = \dim V - \dim N(T)\\
    & \implies \dim W = \dim V - \dim N(T)
\end{align*}

For cases where $\dim V = \dim W$, $\dim N(T)$ can clearly be 0, so
injective linear maps exist for that case. However, if $\dim V > \dim W$,
then $\dim N(T)$ can never be 0, so $N(T) \neq \{\mathbf{0}\}$.

\emptyline
\textit{(b)} False.

As a counterexample, let $V = \R^3$, $W = \R^2$, and $T \st V \to W$
be given by $T(x, y, z) = (x, y)$.

Then there are infinite choices for $z$ for which $T(x, y, z)$ will
map to the same $(x, y)$. For example, $(1, 2, 3)$ and $(1, 2, 4)$
both map to $(1, 2)$.

\emptyline
\textit{(c)} False.

Let $T$ be the linear map defined in (b). Since it maps from a vector
space of dimension 3, the vector space should have basis
$\{\mathbf{x}_1, \mathbf{x}_2, \mathbf{x}_3\}$. However the vector
space it maps to is of dimension 2, so it has basis
$\{\mathbf{x}_1, \mathbf{x}_2\}$, which is of a different cardinality to
the basis of $T$'s domain.

\end{document}