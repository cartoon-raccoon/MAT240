\documentclass[12pt, a4paper]{article}
\setlength{\parindent}{0pt}
\usepackage[a4paper, portrait, margin=1in]{geometry}
\usepackage{preamble}

\newtheorem*{theorem}{Theorem}

\newcommand{\F}{\mathbb{F}}

\begin{document}
\textbf{(Q1)}

\textit{(a)}

\begin{proof}
    We define the two multiplicands as follows:

    \[
        \begin{pmatrix}
            a_{11} & a_{12} & \ldots & a_{1n}\\
            a_{21} & \ddots &        & \vdots\\
            \vdots &        & \ddots & \vdots\\
            a_{m1} & \ldots & \ldots & a_{mn}\\
        \end{pmatrix}
        \begin{pmatrix}
            0_1\\
            0_2\\
            \vdots\\
            1_j\\
            0_{j+1}\\
            \vdots\\
            0_n
        \end{pmatrix}
    \]

    By definition of matrix multiplication, this would evaluate to:
    \[
        \begin{pmatrix}
            0 + 0 + \ldots + a_{1j} + \ldots + 0\\
            0 + 0 + \ldots + a_{2j} + \ldots + 0\\
            0 + 0 + \ldots + a_{3j} + \ldots + 0\\
            \vdots
            0 + 0 + \ldots + a_{nj} + \ldots + 0\\
        \end{pmatrix}
        =
        \begin{pmatrix}
            a_{1j}\\
            a_{2j}\\
            a_{3j}\\
            \vdots\\
            a_{nj}\\
        \end{pmatrix}
    \]
\end{proof}

\textit{(b)}

\begin{proof}
    By definition of matrix multiplication, the $ij$-th entry of the
    product $AB$ of two matrices $A \in \mathcal{M}_{m \times n}(\F)$ and
    $B \in \mathcal{M}_{n \times k}(\F)$ is
    \[
        (AB)_{ij} = \sum_{l = 1}^{n} A_{il}B_{lj}
    \]
    
    Where $B_{lj}$ is the $l$-th entry of the $j$-th column of $B$.

    Then, treating the $j$-th column of $B$ as a matrix and applying
    matrix multiplication to $A\mathbf{b}_j$, we have:
    \[
        A\mathbf{b}_j = \sum_{l = 1}^{n} A_{il}(\mathbf{b}_j)_l
    \]

    Where $(\mathbf{b}_j)_l$ is the $l$-th entry of $\mathbf{b}_j$,
    the $j$-th column of $B$, the same as what was defined for the
    product of $AB$ above.

    Since $j$ is arbitrary, this applies to all columns $1 \ldots k$
    of matrix $B$.
\end{proof}
\end{document}