\documentclass[12pt, a4paper]{article}
\setlength{\parindent}{0pt}
\usepackage[a4paper, portrait, margin=1in]{geometry}
\usepackage{preamble}

\newtheorem*{theorem}{Theorem}

\newcommand{\F}{\mathbb{F}}

\begin{document}
\textbf{(Q3)}

\textit{(a)}

\[
    \begin{pmatrix}
        5 & 6 & 0 & 0 & 0\\
        2 & 3 & 0 & 0 & 0\\
        0 & 0 & 1 & 9 & 4\\
        0 & 0 & 7 & 3 & 2\\
        0 & 0 & 4 & 5 & 8\\
    \end{pmatrix}
\]

\textit{(b)}
\begin{proof}
    Breaking down the two block-diagonal matrices that we are multiplying,
    are of the form:
    \[
        \begin{bmatrix}
            a_{11} & a_{12} & \ldots & a_{1k} & 0_{1(k+1)} & \ldots & 0_{1n}\\
            a_{21} & \ddots &        &        & \vdots     &        &\\
            \ldots &        & \ddots &        & \vdots     &        &\\
            a_{k1} & \ldots &        & a_{kk} & 0_{k(k+1)} & \ldots & 0_{kn}\\
            0_{(k+1)1} & 0_{(k+1)2} & \ldots & 0_{(k+1)k}
            & b_{(k+1)(k+1)} & \ldots & b_{(k+1)n}\\
            \vdots & &        &        & \vdots &            &        &\\
            \ldots &        &        & & \vdots & \ddots &\\
            0_{n1} & \ldots & \ldots & 0_{nk} & b_{n(k+1)} & \ldots & b_{nn}\\
        \end{bmatrix}
    \]

    For the other multiplicand, the $a$ and $b$ entries are replaced
    with $c$ and $d$ respectively.

    From the defintiion of matrix multiplication, we have that for
    any entry $ij$ in the resultant matrix, it is equivalent to
    \[
        (MN)_{ij} = \sum_{l = 1}^{n} M_{il}N_{lj}
    \]

    Considering both $l \leq k$ for both matrices, then for some $l$ we have 
    \[(MN)_{il} = \sum_{q = 1}^{l} a_{iq}c_{ql}\] and
    \[(MN)_{lj} = \sum_{q = 1}^{l} a_{qi}c_{jq}\]

    If there exists one $l > k$, then that entry in one matrix will be
    multiplied by a zero entry in the other matrix
    and so the entry in the resulting matrix will be zero.

    However, if both $l > k$, then for some $l$ we have 
    \[(MN)_{il} = \sum_{q = 1}^{n-k} b_{iq}d_{ql}\] and
    \[(MN)_{lj} = \sum_{q = 1}^{n-k} b_{qi}d_{jq}\]

    and a similar result if there is one $l \leq k$.

    \newpage
    Considering all these cases, we have the following matrix:
    \[
        \begin{bmatrix}
            a_{11}c_{11} & a_{12}c_{12} & \ldots & a_{1k}c_{1k} 
            & 0_{1(k+1)} & \ldots & 0_{1n}\\
            a_{21}c_{21} & \ddots &        &        & \vdots     &        &\\
            \ldots &        & \ddots &        & \vdots     &        &\\
            a_{k1}c_{k1} & \ldots &        & a_{kk}c_{kk} 
            & 0_{k(k+1)} & \ldots & 0_{kn}\\
            0_{(k+1)1} & 0_{(k+1)2} & \ldots & 0_{(k+1)k}
            & bd_{(k+1)(k+1)} & \ldots & bd_{(k+1)n}\\
            \vdots & &        &        & \vdots &            &        &\\
            \ldots &        &        & & \vdots & \ddots &\\
            0_{n1} & \ldots & \ldots & 0_{nk} & bd_{n(k+1)} & \ldots & bd_{nn}\\
        \end{bmatrix}
    \]

    which takes the form
    $\begin{pmatrix}AC & O_{k,(n-k)} \\ O_{(n-k),k} & BD\end{pmatrix}$.
\end{proof}

\textit{(c)}
\begin{proof}
    First assume $M$ is invertible, so
    $\exists N \in \mathcal{M}_{n \times n} \st MN = NM$.

    Let $N = \begin{pmatrix}C&O_{k,(n-k)}\\O_{(n-k),k}&D\end{pmatrix}$.
    $N$ is block diagonal.
    
    Then

    \[
        MN = \begin{pmatrix}
            AC & O_{k,(n-k)}\\
            O_{(n-k),k} & BD
        \end{pmatrix},\quad
        NM = \begin{pmatrix}
            CA & O_{k,(n-k)}\\
            O_{(n-k),k} & DB
        \end{pmatrix}
    \]

    Since $MN = NM$, $AC = CA$ and $BD = DB$ and thus $A$ and $B$
    are invertible.

    Now assume $A$ and $B$ are invertible, so
    $\exists C, D \st AC - CA \text{ and } BD = DB$.
    Then let $N$ be defined as above.

    Then $MN, NM$ are defined by the above equations.
    Since $AC = CA$ and $BD - DB$, $MN = NM$ and thus $M$ is invertible.
\end{proof}
\end{document}