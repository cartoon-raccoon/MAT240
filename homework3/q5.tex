\documentclass[12pt, a4paper]{article}
\setlength{\parindent}{0pt}
\usepackage[a4paper, portrait, margin=1in]{geometry}
\usepackage{preamble}

\newtheorem*{theorem}{Theorem}

\newcommand{\F}{\mathbb{F}}
\newcommand{\im}{\operatorname{im}}

\begin{document}
\textbf{Q(5)}

\textit{(a)}
\begin{proof}
    We aim to prove $T$ is injective using the definition of
    injectivity, that is,
    \[
        T(x_1) = T(x_2) \implies x_1 = x_2
    \]
    First suppose $T(x_1) = T(x_2)$ and $x_1 = x_2$. Then
    $T(x_1) - T(x_2) = \textbf{0}_W$.
    By linearity of $T$, we have
    $T(x_1) - T(x_2) = T(x_1 - x_2) = \textbf{0}_W$. Since
    $x_1 = x_2$, we have

    \[T(x_1 - x_2) = T(\textbf{0}_V) = \textbf{0}_W\]

    So $N(T) = \{\textbf{0}_V\}$, since any other value implies
    $x_1 \neq x_2$.

    Now suppose $N(T) = \{\textbf{0}_W\}$, and let $T(x_1) = T(x_2)$.

    Then $T(x_1) - T(x_2) = \textbf{0}_W$. Again, by linearity of $T$,
    we have $T(x_1) - T(x_2) = T(x_1 - x_2) = \textbf{0}_W$. Since
    $N(T) = \{\textbf{0}_V\}$, $x_1 - x_2$ has to be $\textbf{0}_V$.
    Then 
    
    \[x_1 - x_2 = \textbf{0}_V \implies x_1 = x_2\]

    Which satisfies the definition of injectivity.
\end{proof}

\textit{(b)}
\begin{proof}
    We assume for the entire proof that $\dim V = \dim W = n$.

    First, suppose $T$ is injective; that is, $N(T) = \{\textbf{0}_V\}$.

    Then by the Rank-Nullity Theorem,
    \[
        \dim V - \dim N(T) = n - 0 = n = \dim (\im T)
    \]

    We have $\im T$ is a subspace of $W$, and
    $\dim (\im T) = \dim V = \dim W$, so $\im T = W$, and thus
    $T$ is surjective.

    Now suppose $T$ is surjective, so
    $\im T = W \implies \dim (\im T) = \dim W = n$. Then
    \begin{gather*}
        \dim (\im T) + \dim N(T) = n = \dim W = \dim V\\
        \dim (\im T) = \dim W = n \implies \dim N(T) = 0 \implies N(T)
        = \{\textbf{0}_V\}
    \end{gather*}

    So $T$ is injective.
\end{proof}
\end{document}