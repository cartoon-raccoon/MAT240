\documentclass[12pt, a4paper]{article}
\setlength{\parindent}{0pt}
\usepackage[a4paper, portrait, margin=1in]{geometry}
\usepackage{preamble}

\newtheorem*{theorem}{Theorem}

\newcommand{\F}{\mathbb{F}}

\begin{document}
\textbf{(Q6)}

\textit{(a)}
We obtain a matrix for this linear transformation by taking $T(e_i)$
for each standard basis vector for $\R^4$.
\begin{gather*}
    T(1, 0, 0, 0) = (1, 0, 1, 1)\\
    T(0, 1, 0, 0) = (2, 0, 1, 3)\\
    T(0, 0, 1, 0) = (3, 0, 2, 4)\\
    T(0, 0, 0, 1) = (1, 0, 0, 2)
\end{gather*}

Taking each resultant vector as a row, we obtain the matrix
\[
    \begin{bmatrix}
        1 & 0 & 1 & 1\\
        2 & 0 & 1 & 3\\
        3 & 0 & 2 & 4\\
        1 & 0 & 0 & 2
    \end{bmatrix}
\]

\emptyline
\textit{(b)}

We find $N(T)$ by solving the matrix obtained in \textit a for a
system of homogenous equations. Applying row-reduction, we have:

\[
    \begin{bmatrix}
        1 & 0 & 1 & 1\\
        2 & 0 & 1 & 3\\
        3 & 0 & 2 & 4\\
        1 & 0 & 0 & 2
    \end{bmatrix} \to
    \begin{bmatrix}
        1 & 0 & 1 & 1\\
        0 & 0 & 1 & -1\\
        0 & 0 & -1 & 1\\
        0 & 0 & -1 & 1
    \end{bmatrix} \to
    \begin{bmatrix}
        1 & 0 & 0 & 2\\
        0 & 0 & 1 & -1\\
        0 & 0 & 0 & 0\\
        0 & 0 & 0 & 0
    \end{bmatrix}
\]

Let $y = t$, $w = s$. Then $x = -2s$, $z = s$.
So the general solution has the form
\[(-2s, t, s, s)\]

Grouping by parameter, we then obtain
\[
    t(0, 1, 0, 0) + s(-2, 0, 1, 1)
\]
So a basis for $N(T)$ is $\{(0, 1, 0, 0), (-2, 0, 1, 1)\}$
and has dimension 2.

\emptyline
\textit{(c)} No.

In order for $T$ to be invertible, it has to be both injective
and surjective. By proof of Q5(a), in order for $T$ to be injective
it has to have $N(T) = \{\textbf{0}\}$. Since this is not the case,
$T$ is not injective and thus not invertible.
\end{document}