\documentclass[12pt, a4paper]{article}
\setlength{\parindent}{0pt}
\usepackage[a4paper, portrait, margin=1in]{geometry}
\usepackage{preamble}

\newtheorem*{theorem}{Theorem}

\newcommand{\F}{\mathbb{F}}

\begin{document}

\textbf{(Q2)}

\textit{(a)}

\begin{align*}
    x + 0y - 2z + w & = 0\\
    x + y - z + 0w & = 0\\
    2x + y - 3z + w & = 0
\end{align*}

\textit{(b)}

\begin{gather*}
    \begin{pmatrix}
        1 & 0 & 2 & 1 & \vert & 0\\
        1 & 1 & -1 & 0 & \vert & 0\\
        2 & 1 & -3 & 1 & \vert & 0\\
    \end{pmatrix}\\
    \downarrow\\
    \begin{pmatrix}
        1 & 0 & -2 & 1 & \vert & 0\\
        0 & 1 & 1 & -1 & \vert & 0\\
        0 & 1 & 1 & -1 & \vert & 0\\
    \end{pmatrix}\\
    \downarrow\\
    \begin{pmatrix}
        1 & 0 & -2 & 1 & \vert & 0\\
        0 & 1 & 1 & -1 & \vert & 0\\
        0 & 0 & 0 & 0 & \vert & 0\\
    \end{pmatrix}\\
\end{gather*}

We then have

\begin{align*}
    x & = 2s - t\\
    y & = -s + t\\
    z & = s\\
    w & = t
\end{align*}

And thus
\begin{gather*}
    (2s - t, -s + t, s, t)\\
    = (2s, -s. s. 0) + (-t, t, 0, t)\\
    = s(2, -1, 1, 0) + t(-1, 1, 0, 1)
\end{gather*}

\textit{(c)}

This follows from the fact that $(2, -1, 1, 0)$ and $(-1, 1, 0, 1)$
are linearly independent, and thus is a basis for $W$.

\textit{(d)}

$W$ is closed under scaling and vector addition, so it is a subspace.

\end{document}