\documentclass[12pt, a4paper]{article}
\setlength{\parindent}{0pt}
\usepackage[a4paper, portrait, margin=1in]{geometry}
\usepackage{preamble}

\newtheorem*{theorem}{Theorem}

\newcommand{\F}{\mathbb{F}}

\newcommand{\Fcal}{\mathcal{F}}
\newcommand{\Tcal}{\mathcal{T}}

\begin{document}

\textbf{(Q1)}

\textit{(a)} Since $\Fcal$ is the function space mapping into a vector space,
these properties arise from the scalar and additive properties of $V$.

\textit{(b)}
\begin{proof}
    Set $s \in S$ as arbitrary, and $f, g, h \in \Fcal$. Then
    \begin{align*}
        (f + g)(s) & = f(s) + g(s)\\
        (g + h)(s) & = g(s) + h(s)\\
        f(s) + g(s) + h(s) & = f(s) + (g + h)(s)
    \end{align*}
\end{proof}

\textit{(c)}
\begin{proof}
    Set $s \in S$ as arbitrary, and $f, g \in \Fcal$. Then
    \begin{align*}
        c(f(s)) = (cf)(s), & c(g(s)) = (cg)(s)\\
        (cf)(s) + (cg)(s) & = c(f(s)) + c(g(s))\\
        & = c(f(s) + g(s))\\
        & = c((f + g)(s))
    \end{align*}
\end{proof}

\textit{(d)}
\begin{proof}
    Let $s \in S$ be arbitrary, and $f \in \Fcal$. Then
    \begin{align*}
        (f + \mathbf{0})(s) & = f(s) + \mathbf{0}(s)\\
        & = f(s) + \mathbf{0}_V\\
        & = f(s)
    \end{align*}
\end{proof}

\textit{(e)}
\begin{proof}
    Let $s \in S$, $g = -f$. Then
    \begin{align*}
        (f + g)(s) & = (f - f)(s)\\
        & = f(s) - f(s)\\
        & = \mathbf{0}_V
    \end{align*}
\end{proof}
\end{document}