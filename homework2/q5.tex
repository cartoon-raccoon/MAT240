\documentclass[12pt, a4paper]{article}
\setlength{\parindent}{0pt}
\usepackage[a4paper, portrait, margin=1in]{geometry}
\usepackage{preamble}

\newtheorem*{theorem}{Theorem}

\newcommand{\F}{\mathbb{F}}

\begin{document}

\textbf{(Q5)}

\begin{proof}
    Since this system has a unique solution, it has to be trivial, (i.e. $x, y = 0$)

    We first assume $ad - bc \neq 0$. We then have a matrix
    

    \[
        \begin{pmatrix}
            a & b \\
            c & d \\
        \end{pmatrix}
    \]

    By row-reduction:

    \[
        \begin{pmatrix}
            a & b \\
            c & d \\
        \end{pmatrix}
        \rightarrow
        \begin{pmatrix}
            ad & bd \\
            bc & bd \\
        \end{pmatrix}
        \rightarrow
        \begin{pmatrix}
            ad - bc & 0 \\
            bc & bd \\
        \end{pmatrix}
        \rightarrow
        \begin{pmatrix}
            1 & 0 \\
            bc & bd \\
        \end{pmatrix}
    \]

    Then $x = 0$, and by substitution into the original system $y = 0$.

    We then assume the converse, that there is a trivial solution $x, y = 0$.

    Then $\{(a, c), (b, d)\}$ is linearly independent.

    Then $d(a, c) \neq c(b, d) \implies (ad, ac) \neq (bc, bd)$.
\end{proof}

\end{document}