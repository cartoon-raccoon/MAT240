\documentclass[12pt, a4paper]{article}
\setlength{\parindent}{0pt}
\usepackage[a4paper, portrait, margin=1in]{geometry}
\usepackage{preamble}

\newtheorem*{theorem}{Theorem}

\newcommand{\F}{\mathbb{F}}

\begin{document}

\textbf{(Q10)}

\textit{(a)}

\begin{proof}
    We intend to show that for any two vectors $w_1 \in W_1$ and
    $w_2 \in W_2$,
    $\exists W_2 \sse V \st w_1 + w_2 \in V$ and
    $W_1 \cap W_2 = \{0\}$.

    Let $B$ be the basis of $V$ and $B_1$ the basis of $W_1$,
    and $n = \dim V$.

    Then $B = \{v_1, v_2, \ldots v_n\}$
    and $B_1 = \{u_1, u_2, \ldots u_k\}$. Then by the Replacement Theorem,
    we have

    \[
        B = \{u_1, u_2, \ldots u_k, v_{k + 1}, v_{k + 2}, \ldots v_n\}
    \]

    Let $B_2 = B \setminus B_1 = \{v_{k + 1}, \ldots v_n\}$.
    Since $B$ is a basis, all its constituent vectors are linearly independent
    and thus $B_2 \sse B \implies B_2$ is linearly independent.

    Let $W_2 = \operatorname{span} B_2$. By Theorem 20, $W_2$ is a subspace of $V$.

    Since $B_1 \cap B_2 = \phi$, no vector from $W_1$ can be expressed in
    terms of vectors from $B_2$, thus the only common element $W_1$ and $W_2$
    have is $\mathbf{0}$.
\end{proof}

\textit{(b)}
Since $B_2 = B \setminus B_1$, $|B_2| = |B| - |B_1|$ and thus
$\dim W_2 = \dim V - \dim W_1$.

\textit{(c)} Yes.

\begin{proof}
    Assume $\exists W_3 \st V = W_1 \oplus W_2 = W_1 \oplus W_3$.
    We aim to show that $W_2 = W_3$.

    We then have
    \begin{gather*}
        V = \{v \in V = w_1 + w_2 \st w_1 \in W_1 \text{ and } w_2 \in W_2\}\\
        V = \{v \in V = w_1 + w_3 \st w_1 \in W_1 \text{ and } w_3 \in W_3\}
    \end{gather*}

    Then $\forall v \in V, v - w_1 = w_2$ and $v - w_2 = w_3$.

    Since every vector in $W_2$ has a corresponding vector in $W_3$
    and vice-versa, by mutual subset inclusion, $W_2 = W_3$.
\end{proof}

\end{document}