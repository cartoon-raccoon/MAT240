\documentclass[12pt, a4paper]{article}
\setlength{\parindent}{0pt}
\usepackage[a4paper, portrait, margin=1in]{geometry}
\usepackage{preamble}

\newtheorem*{theorem}{Theorem}

\newcommand{\F}{\mathbb{F}}

\begin{document}

\textbf{(Q9)}

\begin{proof}
    We first assume:

    \[
        \sum_{i = 1}^{k} c_iA_i = 0 \implies \forall i \; c_i = 0
    \]

    Then since $(A^t)_{ij} = A_{ji} \forall A$, the same
    operation essentially takes place with $A^t$, which means

    \[
        \sum_{i = 1}^{k} c_iA^{t}_{k} = 0 \implies \forall i \; c_i = 0
    \]

    The same argument holds for the converse.
\end{proof}

\end{document}