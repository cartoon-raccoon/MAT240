\documentclass[12pt, a4paper]{article}
\setlength{\parindent}{0pt}
\usepackage[a4paper, portrait, margin=1in]{geometry}
\usepackage{preamble}

\newtheorem*{theorem}{Theorem}

\newcommand{\F}{\mathbb{F}}

\begin{document}

\textbf{(Q3)}

\textit{(a)}

\begin{proof}
    Since $W$ is the set of solutions, then
    $\forall w \in W, \; w = \sum_{i = 1}^{k} t_ix_i$.

    Let $S = \{\mathbf{x}_1, \mathbf{x}_2,\ldots \mathbf{x}_k\}$.

    $W$ represents every linear combination of the base vectors
    $\mathbf{x}_1 \ldots \mathbf{x}_k$, so $W = \operatorname{span} S$.
\end{proof}

\textit{(b)}

\begin{proof}
    For the sake of contradiction, assume that
    \[
        \sum_{i = i}^{k} t_ix_i = \mathbf{0} \text{ and }
        \exists t_j \st t_j \neq 0
    \]

    If there is only one such $t$, then that $t$ has to be zero.

    If there is more than one, then $S$ no longer spans $w$.

    Thus, $S$ has to be linearly independent.
\end{proof}

\textit{(c)}

Since $\{\mathbf{x}_1, \mathbf{x}_2,\ldots \mathbf{x}_k\}$ is linearly
independent and spans $W$ is is a basis for $W$.

\end{document}