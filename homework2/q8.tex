\documentclass[12pt, a4paper]{article}
\setlength{\parindent}{0pt}
\usepackage[a4paper, portrait, margin=1in]{geometry}
\usepackage{preamble}

\newtheorem*{theorem}{Theorem}

\newcommand{\F}{\mathbb{F}}

\begin{document}

\textbf{(Q8)}

\textit{(a)}

Let $A \in \mathbf{Sym}_n(\F)$. Then $A$ can be expressed in the form

\[
    \begin{pmatrix}
        a_{11} & a_{12} & \ldots & a_{1n}\\
        a_{12} & \ldots & \ldots & \ldots\\
        \ldots &        &        & \\
        a_{1n} &        &        & a_{nn}\\
    \end{pmatrix}
\]

Then $cA$ for some $c \in \F$ will evaluate to:

\[
    \begin{pmatrix}
        ca_{11} & ca_{12} & \ldots & ca_{1n}\\
        ca_{12} & \ldots & \ldots & \ldots\\
        \ldots &        &        & \\
        ca_{1n} &        &        & ca_{nn}\\
    \end{pmatrix}
\]

Which is still symmetrical.

Similarly, $A + B$ where $B \in \mathbf{Sym}_n(\F)$ evaluates to:

\[
    \begin{pmatrix}
        a_{11} + b_{11} & a_{12} + b_{12} & \ldots & a_{1n} + b_{1n}\\
        a_{12} + b_{12} & \ldots & \ldots & \ldots\\
        \ldots &        &        & \\
        a_{1n} + b_{1n} &        &        & a_{nn} + b_{nn}\\
    \end{pmatrix}
\]

Which is also symmetrical.

\textit{(b)}

Extending from Week 4's lecture activities on skew-symmetric matrices,
we can let the basis for $\mathbf{Sym}_n(\F)$ be

\[
    \left\{E_{ij} + E_{ji} \st i < j \right\} 
    \cup \left\{E_{ij} \st i = j\right\}
\]

\textit{(c)}

Again extending from week 4's activities, $\dim \mathbf{Sym}_n(\F) = \frac{n(n - 1)}{2} + n$.

\end{document}