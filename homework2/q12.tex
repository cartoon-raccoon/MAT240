\documentclass[12pt, a4paper]{article}
\setlength{\parindent}{0pt}
\usepackage[a4paper, portrait, margin=1in]{geometry}
\usepackage{preamble}

\newtheorem*{theorem}{Theorem}

\newcommand{\F}{\mathbb{F}}

\begin{document}

\textbf{(Q12)}

\textit{(a)} True.

\begin{proof}
    By definition of vector space sums,
    \[
        W_1 + W_2 = \{w = w_1 + w_2 \st w_1 \in W_1, w_2 \in W_2\}
    \]

    Since $W_1 + W_2 = W_1$, for all $w \in W_1 + W_2, w \in W_1$, then
    $w_1, w_2 \in W_1$ and thus $W_2 \sse W_1$.
\end{proof}

\textit{(b)} True.

\begin{proof}
    Rearrange the set into a system of homogeneous equations. If there
    is one solution, then it has to be unique and thus trivial.

    If there is more than one (nontrivial) solution, it can be expressed in the form

    \[
        \sum_{i = 1}^{k} t_i \mathbf{x}_i
    \]

    Where $\mathbf{x}$ is a basic solution and $t_i$ is an arbitrary
    field element.

    Since $\F$ is infinite, there are infinite choices for $t_i$, and
    thus there are infinite solutions.
\end{proof}

\textit{(c)} False.

Let $V = P_3(\F)$ and $S_1 = \{cx\}$ where $c \in \F$, $S_2 = \{x, x^2, x^3\}$.

$\operatorname{span} S_1$ is clearly a subset of $\operatorname{span} S_2$,
but $S_1$ is clearly not a subset of $S_2$.

\end{document}