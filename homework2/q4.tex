\documentclass[12pt, a4paper]{article}
\setlength{\parindent}{0pt}
\usepackage[a4paper, portrait, margin=1in]{geometry}
\usepackage{preamble}

\newtheorem*{theorem}{Theorem}

\newcommand{\F}{\mathbb{F}}

\begin{document}

\textbf{(Q4)}

Assuming homogeneity and applying row reduction, we see the following steps:

\begin{gather*}
    \begin{pmatrix}
        1 & 2 & 5 & \vert & 0\\
        -1 & 0 & 5 & \vert & 0\\
        0 & 1 & 5 & \vert & 0\\
        1 & 2 & 5 & \vert & 0
    \end{pmatrix}\\
    \downarrow\\
    \begin{pmatrix}
        1 & 2 & 5 & \vert & 0\\
        0 & 2 & 10 & \vert & 0\\
        0 & 1 & 5 & \vert & 0\\
        1 & 2 & 5 & \vert & 0
    \end{pmatrix}\\
    \downarrow\\
    \begin{pmatrix}
        1 & 2 & 5 & \vert & 0\\
        1 & 2 & 5 & \vert & 0\\
        0 & 2 & 10 & \vert & 0\\
        0 & 1 & 5 & \vert & 0
    \end{pmatrix}\\
    \downarrow\\
    \begin{pmatrix}
        1 & 2 & 5 & \vert & 0\\
        0 & 0 & 0 & \vert & 0\\
        0 & 2 & 10 & \vert & 0\\
        0 & 1 & 5 & \vert & 0
    \end{pmatrix}\\
    \downarrow\\
    \begin{pmatrix}
        1 & 2 & -5 & \vert & 0\\
        0 & 1 & 5 & \vert & 0\\
        0 & 0 & 0 & \vert & 0\\
        0 & 0 & 0 & \vert & 0
    \end{pmatrix}\\
\end{gather*}

The final set of equations forms a set that is not linearly independent.

\end{document}