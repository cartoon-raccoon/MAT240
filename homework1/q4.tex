\documentclass[12pt, a4paper]{article}
\setlength{\parindent}{0pt}
\usepackage[a4paper, portrait, margin=1in]{geometry}
\usepackage{preamble}

\newtheorem*{theorem}{Theorem}

\newcommand{\F}{\mathbb{F}}
\newcommand{\divs}{\;|\;}

\begin{document}

\textbf{(Q4)}

\textit{(a)}

\begin{proof}
    By earlier proof, 
    $\gcd (a, n) = 1 \implies \exists x \in \Z 
    \st [a] \cdot [x] = [1] \text{ in } \Z_n$.

    We now seek to prove the converse:

    \[
        \exists x \in \Z_n \st [a] \cdot [x] = [1] \implies
        \gcd (a, n) = 1
    \]

    Let $[a] \cdot [x] = [1]$ in $\Z_n$. Then by definition of 
    congruence modulo $n$, we have
    
    \[
        ax \equiv 1 \pmod n \implies n \divs 1 - ax
        \implies \exists y \in \Z \st yn = 1 - ax
    \]

    Then by B\'{e}zout's Identity:

    \[
        \exists x, y \in \Z \st ax + yn = 1 \implies \gcd (a, n) = 1
    \]
\end{proof}

Since both directions have been proven, this statement is if and only if.

\textit{(b)} Since $\forall n \in \N, \Z_{n}$ fulfills all the field axioms
except for that of the multiplicative identity, if every nonzero element of
$\Z_n$ has a multiplicative identity, $\Z_n$ is a field. This occurs iff
$n$ is prime.

\end{document}