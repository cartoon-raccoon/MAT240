\documentclass[12pt, a4paper]{article}
\setlength{\parindent}{0pt}
\usepackage[a4paper, portrait, margin=1in]{geometry}
\usepackage{preamble}

\newtheorem*{theorem}{Theorem}

\newcommand{\F}{\mathbb{F}}

\begin{document}

\textbf{(Q9)}

\begin{proof}
    By the Fundamental Theorem of Algebra, any polynomial $p$ over $\C$ can
    be reduced into a product of linear terms over $\C$, the number of terms
    being $\deg p$. Additionally, if $a$ is a root of $p$, then $\bar{a}$
    is also a root of $p$. Thus, roots of $p$ come in conjugate pairs.

    What is left is to consider specific cases of $\deg p$.

    \vspace{5mm}
    \textbf{Considering cases where $\deg p = 1$:}

    Suppose for the sake of contradiction, that $p$ is reducible. Then

    \[
        \exists a, b \in P(\R) \st ab - p \implies \deg a + \deg b = \deg p
    \]

    It is also a condition that $\deg a, \deg b \neq 0$.

    However, $\deg p = 1 \implies \deg a$ or $\deg b = 0$, which is a contradiction.

    \vspace{5mm}
    \textbf{Considering cases where $\deg p = 2$:}

    By earlier proof, if $p$ is irreducible, then it has no real roots.

    \vspace{5mm}
    \textbf{Considering cases for $\deg p$ is even and $> 2$:}

    Any polynomial of an even degree can be expressed as a product of
    polynomials of degree 2, so it is reducible.

    \vspace{5mm}
    \textbf{Considering cases for $\deg p$ is odd and $> 1$:}

    Since roots of polynomials come in conjugate pairs, there will always
    be one root remaining. Since roots come in conjugate pairs, this root has to
    be equal to its conjugate:

    \[
        a = \bar{a} \implies a = x + 0i
    \]

    Since the imaginary part of the root is $0$, it is wholly real.

    Since $p$ has a real root $a$, $\exists q \in P(\R) \st p(x) = (x - a)p(x)$,
    thus $p$ is reducible.

\end{proof}
\end{document}