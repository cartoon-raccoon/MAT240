\documentclass[12pt, a4paper]{article}
\setlength{\parindent}{0pt}
\usepackage[a4paper, portrait, margin=1in]{geometry}
\usepackage{preamble}

\newtheorem*{theorem}{Theorem}

\newcommand{\F}{\mathbb{F}}
\newcommand{\fchar}{\operatorname{char}}

\begin{document}

\textbf{(Q6)}

\textit{(a)}

\begin{proof}
    We have that for any $\Z_n$:

    \[
        \underbrace{[1] + [1] + \ldots + [1] + [1]}_{x \text{ times}} = [x]
    \]

    Then in $\Z_p$:

    \[
        \underbrace{[1] + [1] + \ldots + [1] + [1]}_{p \text{ times}} = [p] = [0]
    \]

    Which means $\fchar (\Z_p) = p$.
\end{proof}

\textit{(b)}

\begin{proof}
    Let $\F$ be a field. Suppose for the sake of contradiction that
    $p = \fchar (\F) \neq 0$ and $p$ is composite. Then

    \[
        \exists x, y \in \Z \st xy = \fchar (\F)
        \implies xy \cdot 1_{\F} = (x \cdot 1_{F}) \cdot (y \cdot 1_{\F})
        = 0
    \]

    which in turn means $x \cdot 1_{\F} = 0$ or $y \cdot 1_{\F} = 0$.

    Since $x, y < p$, this contradicts the assumption that $p = \fchar (\F)$,
    so $p$ has to be prime.

    For fields that have characteristic 0, consider infinite fields such as $\Q$
    or $\R$. In such fields, repeatedly adding 1 to itself will never yield
    0, thus fields can have characteristic 0. If $\fchar (\F) \neq 0$, then
    $\fchar (\F)$ is prime.
\end{proof}

\end{document}