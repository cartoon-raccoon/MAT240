\documentclass[12pt, a4paper]{article}
\setlength{\parindent}{0pt}
\usepackage[a4paper, portrait, margin=1in]{geometry}
\usepackage{preamble}

\newtheorem*{theorem}{Theorem}

\newcommand{\F}{\mathbb{F}}

\begin{document}
\textbf{(Q5)}

\textit{(a)}
\begin{proof}
    By definition of symmetry, an entry in row $i$ column $j$
    should equal the entry in row $j$ column $i$.

    This is a result of the Axiom of Commutativity.
\end{proof}

\textit{(b)}
\begin{proof}
    Given the layout of the multiplication table, this means that
    \[
        \forall a \in \F, (a \cdot 0) = (0 \cdot a) = 0
    \]

    We can prove this as follows:
    \begin{align*}
        \text{(Add. Iden.)} \quad a \cdot 0 & = (a \cdot 0) + 0\\
        \text{(Add. Inv.)} \quad & = (a \cdot 0) + (a \cdot 0 + (-a) \cdot 0)\\
        \text{(Assoc.)} \quad & = (a \cdot 0 + a \cdot 0) + ((-a) \cdot 0)\\
        \text{(Dist. )} \quad & = a \cdot (0 + 0) + (-a) \cdot 0\\
        \text{(Add. Iden.)} \quad & = (0 \cdot a) + (0 \cdot (-a))\\
        \text{(Add. Inv.)} \quad & = 0
    \end{align*}
\end{proof}

\textit{(c)}
This is a result of the uniqueness of each field element:

\[
a + b = c + b \implies a = c
\]

\begin{proof}
    For some $a, b, c \in \F$, we have:
    \begin{align*}
        a + b = c + b & \implies a + b + (-b) = c + b + (-b)\\
        & \implies a + 0 = c + 0\\
        & \implies a = c
    \end{align*}
\end{proof}

\textit{(d)}
This is also a result of the uniqueness of each field element:

\[
    a \cdot c = b \cdot c \implies a = b
\]

\begin{proof}
    For some $a, b, c \in \F \setminus \{0\}$, we have:
    \begin{align*}
        a \cdot b = c \cdot b & \implies 
        a \cdot b \cdot b^{-1} = c \cdot b \cdot b^{-1}\\
        & \implies a \cdot 1 = c \cdot 1\\
        & \implies a = c
    \end{align*}
\end{proof}

\textit{(e)} This set is not a field, as the properties addressed in
\textit{(a), (c)} and \textit{(d)} are violated.

\end{document}