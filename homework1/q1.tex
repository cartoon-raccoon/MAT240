\documentclass[12pt, a4paper]{article}
\setlength{\parindent}{0pt}
\usepackage[a4paper, portrait, margin=1in]{geometry}
\usepackage{preamble}

\newtheorem*{theorem}{Theorem}

\newcommand{\F}{\mathbb{F}}

\begin{document}

\textbf{(Q1)}
\begin{theorem}
    Let $\F$ be a field. Then if $a \in \F \setminus \{0\}$, then
    $a^{-1}$ is invertible and $(a^{-1})^{-1} = a$.
\end{theorem}

\begin{proof}
    Since $a \neq 0_{\F}$, $a^{-1}$ exists in $\F$. Then:

    \begin{align*}
        a & = 1 \cdot a\\
        & = [(a^{-1})^{-1} \cdot a^{-1}] \cdot a\\
        & = (a^{-1})^{-1} \cdot a^{-1} \cdot a\\
        & = (a^{-1})^{-1} \cdot 1_{\F}\\
        & = (a^{-1})^{-1}
    \end{align*}
\end{proof}

\end{document}