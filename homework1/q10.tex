\documentclass[12pt, a4paper]{article}
\setlength{\parindent}{0pt}
\usepackage[a4paper, portrait, margin=1in]{geometry}
\usepackage{preamble}

\newtheorem*{theorem}{Theorem}

\newcommand{\F}{\mathbb{F}}

\begin{document}

\textbf{(Q10)}

\textit{(a)} False.

As a counterexample, let $a = -a$ and $b = -b$. Then $a + a = 0 = b + b$,
while $a$ and $b$ are not necessarily equal.

\textit{(b)} False.

Let $\F$ be a field with the following addition table:

\begin{center}
    \begin{tabular}{ |c|c|c|c|c| }
        \hline
        + & 0 & 1 & a & b \\
        \hline
        0 & 0 & 1 & a & b \\
        \hline
        1 & 1 & 0 & b & a \\
        \hline
        a & a & b & 0 & 1 \\
        \hline
        b & b & a & 1 & 0 \\
        \hline
    \end{tabular}
\end{center}

In this field, $\operatorname{char} (\F)$ = 2, which is prime, but $\F$ is not $\Z_2$.

\textit{(c)} False.

Let $p(x) = (x^2 + 1)^2 = x^4 + 2x^2 + 1$ in $\R$. $p$ is reducible, but has no solutions
in $\R$.

\textit{(d)} True.

\begin{proof}
    Suppose for the sake of contradiction that $p$ is irreducible
    and $p$ has solutions. Then $\exists a \st p(x) = (x - a)q(x)$,
    where $q$ is another polynomial.

    Then $\deg p > 1 \implies \deg q \geq 1$,
    which implies that $p$ is in fact reducible, thus forming a contradiction.
\end{proof}

\end{document}