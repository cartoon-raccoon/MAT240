\documentclass[12pt, a4paper]{article}
\setlength{\parindent}{0pt}
\usepackage[a4paper, portrait, margin=1in]{geometry}
\usepackage{preamble}

\newtheorem*{theorem}{Theorem}

\newcommand{\F}{\mathbb{F}}

\begin{document}

\textbf{(Q3)}

\vspace{5mm}
\textit{(a)} $[5] \cdot [8] \text{ in } \Z_{11} = [40 - (11 \cdot 3)] = [7]$

\vspace{5mm}
\textit{(b)} $[5]^{-1} \text{ in } \Z_{11} = [9]$

\vspace{5mm}
\textit{(c)} $[2] \cdot [3] + 4 \text{ in } \Z_{5} = [6] + [4] = [10] = [0]$

\vspace{5mm}
\textit{(d)} $[6] \cdot [5] \text{ in } \Z_{15} = [30 - (15 \cdot 2)] = [0]$
\end{document}